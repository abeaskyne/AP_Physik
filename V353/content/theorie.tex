\section{Theorie}
\label{sec:Theorie}
\subsection{allgemeine Relaxationsgleichung und praktisches Beispiel des RC-Kreises}
Relaxation beschreibt die  Effekte welche beim Entfernen eines Systems aus seinem oszillierenden Zustands und seiner anschließenden Rückkehr (ohne Oszillation) in 
selbigen Zustand zustande kommen.Im allgemeinen gilt dabei 
\begin{equation}
    \frac{dA}{dt}=c[A(t)-A(\infty)]\,
\end{equation}
bzw. nach Integration und Auflösung nach A(t)
\begin{equation}
    A(t)=A(\infty) + [A(0)-A(\infty)]\cdot e^(ct)\,
\label{eqn:voreqn2}
\end{equation}



Ein praktisches Beispiel für einen Relaxationsvorgang stellen die Auf- und Entladung eines Kodensators im RC-Kreis dar. Aus dem Zusammenhang
\begin{equation}
     U_{C}= \frac{Q}{C}\,
    \label{eqn:eqn2}
\end{equation}
folgt mit dem Ohmschen Gesetz $I= \frac{U_{C}}{R}$ und der Ladungsänderung \begin{equation}
    dQ = Idt
    \label{eqn:eqn3}
\end{equation} 
das $ \frac{dQ}{dt} = (-)\frac{1}{RC}\cdot Q(t)$ gilt.
Mittels Integration folgt dann unter Berücksichtigung der Randbedingung $Q(\infty)=0$
\begin{equation}
Q(t)= Q(0)\cdot exp(-\frac{t}{RC})\,
\label{eqn:eq1}
\end{equation}
DIe Randbedingung folgt dabei aus der Tatsache das der Kondensator sich asymptotisch gegen 0 nähert für t gegen unendlich.
Für die Aufladefunktion des Kondensators bei Verbindung mit einer Spannungsquelle folgt mit ähnlichem Ansatz\\
$Q(t)=C U_{0}(1-exp(\frac{-t}{RC}))$\\
Die Größe RC wird als Zeitkonstante $\tau$ bezeichnet. Innerhalb eines dieser Zeitkonstanten ändert sich die Ladung um den Faktor $\frac{1}{e} \approx \qty{36.8}{\percent}$.
\subsection{Relaxations bei periodischer Auslenkung aus Gleichgewichtslage Anhand eines RC-Kreise}
In diesem Fall des RC-Kreise wird eine Äußere Spannung mit $U(t)=U_{0}\cdot cos(\omega t)$ betrachtet.
Für kleine Frequenzen $\omega$ ($\omega << \frac{1}{RC})$ ist die Kodensatorspannung $U_{c}$ gleich der äußeren Spannung $U_0$. Da der Auf- und Entladeprozess jedoch zeitlich 
durch die Zeitkonstante in seiner Dauer nicht beliebig schnell werden kann, ergibt sich bei steigender Frequenz ein Phasenunterschied zwischen $U_0$ und $U_C$. 
Da desweiteren die Äußere Spannung ein anderes Vorzeichen als die Kodensatorspannung hat, sinkt die Amplitude der Spannung am Kodensator.\\
Mathematisch kann dieses Problem folgender Weise betrachtet werden:\\
Als Grundansatz wählt man
\begin{equation}
    U_{C}(t)= A(\omega)cos(\omega t+ \varphi (\omega))  \,
    \label{eqn:ansatz}
\end{equation}
Mit dem zweiten Kirchhoffschen Gesetz ergibt sich
\begin{equation}
U_{0}cos(\omega t)= I(t)R + A(\omega)cos(\omega t+ \varphi (\omega)) \,
\label{eqn:eqn6}
\end{equation}
Aus \ref{eqn:eqn2} und \ref{eqn:eqn3} folgt dann 
\begin{equation}
I(t)= C\frac{dU_{C}}{dt}
\label{eqn:eqn7}
\end{equation}
Insgesamt ergibt sich dann durch einsetzen von \ref{eqn:eqn7} in \ref{eqn:eqn6}
\begin{equation}
    U_{0}cos(\omega t)= A \omega RC sin(\omega t + \varphi) + A(\omega)cos(\omega t+ \varphi (\omega)) \,
\label{eqn:eqn8}
\end{equation}
Aus der Gültigkeit dieser Gleichung lässt sich durch die Auswahl spezieller werte für t nach einigen Umformungen der Zusammenhang 
\begin{equation}
    \varphi (\omega) = arctan(-\omega RC)\,
    \label{eqn:phase1}
\end{equation}
für die Phasenverschiebung zwischen $U_{C}$ und $U_{0}$ herstellen.\\
Ebenfalls aus \ref{eqn:eqn8} folgt nach Umformung 
\begin{equation}
    A(\omega)= -\frac{sin(\varphi)}{\omega RC}\cdot U_{0}\,
    \label{eqn:amplitude1}
\end{equation}
oder durch verwendung von \ref{eqn:phase1}
\begin{equation}
    A(\omega)= \frac{U_{0}}{\sqrt{1+ (\omega RC)^2}}\,
    \label{eqn:amplitude2}
\end{equation}
als Zusammenhang zwischen der Amplitude von $U_{C}$.\\
Aus \ref{eqn:phase1} lässt sich erkenn, das die Phasendifferenz für kleine Frequenzen gegen 0 und für hohe Frequenzen gegen $\frac{\pi}{2}$ geht.\\
Für die Amplitude lässt sich aus \ref{eqn:amplitude2} leicht erkennen, dass $A$ für kleine Frequenzen gegen $U_{0}$ und für große Frequenzen gegen 0 geht.


\label{sec:Theorie}



\subsection{Integrationsverhalten des RC-Kreises}
Ein RC-Kreis, oder Tiefpass, kann als integrator dienen, wenn die Bedingung $\omega >> \frac{1}{RC}$ erfüllt ist. Es kann die proportionalität von $U_c$ zu $\int U(t) dt$ gezeit werden.
\begin{equation}
    U(t) = U_{R}(t) + U_C(t) = I(t) \cdot R + U_c(t)
    \label{eq:e9} %ZAHL MUSS NOCH ANGEPASST WERDEN
\end{equation}

Mit \ref{eqn:eqn7} wird I(t) ersetzt:\\

\begin{equation}
    U(t) = RC\frac{dU_C}{dt}
    \label{eq:e10} %ZAHL MUSS NOCH ANGEPASST WERDEN
\end{equation}

Wegen der Bedingung $\omega >> \frac{1}{RC}$ folgt dann:
\begin{equation}
    U_C(t) = \frac{1}{RC}\int_{0}^{t}U(t') dt'
    \label{eq:e11} %ZAHL MUSS NOCH ANGEPASST WERDEN
\end{equation}



 Theorie entnommen aus \cite{V353}
