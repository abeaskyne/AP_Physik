\section{Theorie}
\label{sec:Theorie}



\subsection{Integrationsverhalten des RC-Kreises}
Ein RC-Kreis, oder Tiefpass, kann als integrator dienen, wenn die Bedingung $\omega >> \frac{1}{RC}$ erfüllt ist. Es kann die proportionalität von $U_c$ zu $\int U(t) dt$ gezeit werden.
\begin{equation}
    U(t) = U_{R}(t) + U_C(t) = I(t) \cdot R + U_c(t)
    \label{eq:e9} %ZAHL MUSS NOCH ANGEPASST WERDEN
\end{equation}

Mit \ref{} wird I(t) ersetzt:\\

\begin{equation}
    U(t) = RC\frac{dU_C}{dt}
    \label{eq:e10} %ZAHL MUSS NOCH ANGEPASST WERDEN
\end{equation}

Wegen der Bedingung $\omega >> \frac{1}{RC}$ folgt dann:
\begin{equation}
    U_C(t) = \frac{1}{RC}\int_{0}^{t}U(t') dt'
    \label{eq:e11} %ZAHL MUSS NOCH ANGEPASST WERDEN
\end{equation}



\cite{sample}
