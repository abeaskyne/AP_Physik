\section{Diskussion}
\label{sec:Diskussion}
\subsection{Bestimmung von RC mithilfe der Entladekurve}

Die Messwerte die aus der Entladekurve abgelesen wurden, haben einen relativ kleinen Fehler, da dieser über die ganze Messung gleich bleibt, fällt er bei den niedrigeren Werten 
stärker ins Gewicht. Die Messwerte unterliegen aufgrund des Ablesens von Hand und mit Augenmaß einer natürlichen Ungenauigkeit. Im Graphen der Messwerte ist zu erkennen, dass sie jedoch ziemlich nah an der Trendlinie liegen.
Aus den sich aus dem aus dem Fit ergebenden Parametern lässt sich problemlos die Zeitkonstante $RC = 0.00178\pm0.00004$ bestimmen.

\subsection{Bestimmung von RC mithilfe der relativen Amplitude und der Frequenz}

Auch bei dieser Messreihe ließ sich problemlos $RC = 0.00155\pm0.00007$ bestimmen, auch wenn die Messwerte hier einer erhöhten Unsicherheit unterliegen, da dass Ablesen am gerät durch immer kleinere Werte schwieriger und ungenauer wurde.
Aus dem Fit ließ sich problemos der RC-wert bestimmen, auch wenn die Kurve die Messwerte nur mit weniger guter Genauigkeit approximiert. Im Vergleich mit dem Graphen derselben Funktion aber mit dem vorher errechneten $RC = 0.00178\pm0.00004$
lässt sich erkennen dass es im Grunde keine Abweichung zwischen den beiden Kurven gibt die sich nicht durch einfache Messfehler und ungenauigkeiten erklären ließe. Eine fehlerquelle könnte außerdem sein, dass der Innenwiderstand nicht mit eingerechnet wurde, dieser könnte die Amplitude beeinflussen.

\subsection{Bestimmung von RC mithilfe der Phasenverschiebung und der Frequenz}

Die Messwerte der Phasenverschiebung abhängig von der Frequenz haben sehr große Fehler, ausserdem entsprechen sie im geplotteten Graphen nicht dem erwareteten Verlauf von \ref{eqn:phase1} und \ref{eqn:amplitude1}, die ersten beiden Messwerte sind sehr weit von dem Graphen der Funktion entfernt und haben auch ein gegensätzliches Verhalten, die anderen messwerte sind so ungenau und verstreut, dass insgesamt eine RC-berechnung hier nicht möglich war,
mögliche Gründe für die starken Abweichungen und das fehlerhafte Verhalten könnten die extrem ungenauen Ablesemethoden sein. Das ablesen auf dem Oszilloskop war extrem schwierig, aufgrund der kleinen Werte und den wechselnden skalen, außerdem haben sich die Kurven mit der Zeit ab und zu verschoben, und das genaue Ablesen wurde dadurch erschwert. Andere schwierigkeiten waren technischer Natur, so ging das Oszilloskop mehrfach einfach aus, oder die Kurven sprangen hin und her, nach einer gewissen Zeit, und bei hohen frequenzen hat das Oszilloskop überhaupt nichts mehr angezeigt beziehungsweise nur sporadische anscheinend zufällige Kurven.
Ein weiterer Grund könnte sein, dass vorallem im Bereich großer Frequenzen gemessen wurde, bei denen die skalen nicht mehr ausgereicht haben um genaue Ergebnisse abzulesen.
Es wäre auch möglich, dass die ersten messwerte nicht von Extremum zu Extremum abgelesen wurden, das könnte eine hohe Abweichung hervorrufen.

\subsection{Verifikation der Integratorfunktion des RC-Kreises}

Da die Werte der Phasenverschiebung so unzuverlässig und nahezu nicht zu gebrauchen sind, ist auch auf dem polarplot nichts verwertbares erkennbar, im Vergleich mit dem dargestellten Graphen lässt sich kaum übereinstimmung erkennen. Die am Oszilloskop aufgenommenen Bilder zeigen allerdings relativ eindeutig dass die eingegebene Spannung
vom Tiefpass integriert wird.




