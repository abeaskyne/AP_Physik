\input{header.tex}

\subject{V353}
\title{Das Relaxionsverhalten eines RC-Kreises}
\date{%
  Durchführung: 28.11-2023
  \hspace{3em}
  Abgabe: 4.12.2023
}

\begin{document}

\maketitle
\thispagestyle{empty}
\tableofcontents
\newpage

\section{Theorie}
\subsection{Allgemeine Grundlagen und homogenes Magnetfeld eines Helmholzspulenpaares}
Die kleinste Objekt welches ein Magnetfeld erzeugt ist ein magnetischer Dipol, anders als bei elektrischen Feldern bei denen auch Monopole existieren. In der Praxis können
diese Dipole als Permanentmagneten oder Stromdurchflossene Leiter dargestellt werden. Für letztere lässt sich das magnetische Moment leicht berechnen.
Für das Magnetische Moment von Permanentmagneten existiert jedoch kein einfacher mathematischer Zusammenhang zur Berechnung. Stattdessen kann man sich eine Eigenschaft des Magnetischen moments 
zur experimentellen Bestimmung zu nutzen machen. In homogenen Magnetischen Feldern wirkt ein Drehmoment \\

\begin{centering}

   $ \vec{D} = \vec{\mu} \times \vec{B}$ %% kreuzprodukt und begriffe als vektor 

\end{centering}.
\\
Dabei dreht sich der Dipol bis $\mu$ und $B$ gleichgerichtet sind.
zur reallen Umsetzung eines homogenen Magnetischen Feldes können in der Praxis zwei in gleicher Richtung vom gleichen Strom durchflossene Spulen genutzt werden. Der Abstand 
der Spulen entspricht dabei ihrem Radius R. Das Feld in der Symmetrieachse der Spulen ist dann homogen und es ergibt sich dann mit Bio-Savart zu \\

\begin{centering}

$B(x)= \frac{\mu_{0}I}{2}\frac{R^2}{(R^2 + x^2)^{(5/2)}}$

\end{centering}.\\

Da in Praxis der Abstand der Spulen meist geringfügige Abweichungen vom Spulen Radius hat, folgt hier für das Feld in der Mitte der Spulen\\
\begin{equation}
 B(0)=\frac{\mu_{0}IR^2}{(R^2 + x^2)^{(3/2)}}
 \label{eqn:Bio1}
\end{equation}\\
für hinreichend  kleine Abweichungen von SPulenabstand d und Radius R gilt dann, das das Magnetfeld auf der Symmetrieachse Näherungsweise homogen angenommen werden kann.
Im folgenden werden die theoretischen Grundlagen zu diesen experimentellen Bestimmungsweisen erläutert.
\subsection{Bestimmung über Gravitation}
Auf eine mit dem Permanentmagneten verbundene Masse wirkt die Gravitationskraft, welche eine Drehmoment verursacht. Dem entgegen wirkt das magnetische Feld und das
dadurch entstehende Drehmoment. Für eine bestimmte Feldstärke B baut sich dabei ein Gleichgewicht auf. Es folgt daraus
\begin{equation}
    \mu_{Dipol} \cdot B = m \cdot r \cdot g 
    \label{eqn:grav2} 
\end{equation}.

\subsection{Bestimmung über Schwingunsdauer}
In einem homogenen Magntefeld (z.B. im Feld einer Helmholzspule) verhält sich der magnetische Dipol in der Kugel bei durch Auslenkung aus der Ruhelage erzeugter Schwingung
wie ein harmonischer Oszillator. \\

\begin{centering}

$-\lvert{\vec{\mu}_{\symup{Dipol}} \times \vec{B}}\rvert = J_{\symup{K}} \cdot \frac{d^2 \theta}{dt^2}$

\end{centering}
\\
beschreibt dann die Bewegungsgleichung des Problems. Als Lösung folgt dann die Beziehung\\

\begin{equation}
    \mu_{Dipol}=\frac{4\pi^2 J_{K}}{T^2 B}
    \label{eqn:schwingung1}
\end{equation}
\\


\subsection{Bestimmung durch Präzession}
Wenn auf die Drehachse eines in sich in Rotation befindlichen Körpers eine (auslenkende) Kraft wirkt, stellt sich eine Bewegung der Drehachse um den Drehimpulsvektor ein. 
Diese Art der Bewegung heißt Präzession und entsteht bei der Billardkugel durch eine Auslenkung bei Rotation, wobei die Rotation selbst für eine Stabilität der Bewegung sorgt.
Für die Bewegung der Kugel gilt dann\\

\begin{centering}

$\vec{\mu}_{Dipol} \times \vec{B}= \frac{d\vec{L}_{K}}{dt}$

\end{centering}

%% PRäzessionsfrequenz vllt einbauen
durch Ersetzung des Drehimpuls $L_{K}$ durch das Trägheitsmoment $J_{K}$ und durch die Umlaufzeit $T_{p}$ folgt dann 
\begin{equation}
    \frac{1}{T_{p}}= \frac{\mu_{Dipol} B}{2L_{K}\pi}
    \label{eqn:prz1}
\end{equation}







\label{sec:Theorie}


\section{Durchführung}
\label{sec:Durchführung}
\subsection{Aufbau}
Im Versuch ist ein Spannungsgenerator gegeben, der für die erste Messreihe nur über einen Tiefpass mit einem Oszilloskop verbunden ist. 
Für die Aufnahme der zweiten Messreihe und zur messung zur Bestätigung der Integratorfunktion wird auch noch eine direkte Verbindung zwischen dem Spannungsgenerator und dem Oszilloskop eingeführt.
\subsection{Messreihe zur Entladekurve}
In der ersten messreihe werden Wertepaare der Spannung $U_c$ und der Zeit $t$ von der auf dem oszilloskop dargestellten Entladekurve abgelesen. 
Die Frequenz bleibt dabei konstant auf $77.2\unit{\hertz}$.
\subsection{Messreihe der Phasenverschiebung und der Spannungsamplitude}
In der zweiten Messreihe werden die werte für die Phasenverschiebung zwischen der Generator- und der Kondensatorspannung und die Spannungsamplitude der Kondensatorspannung bei einer variierenden Frequenz $F$ 
von den auf dem oszilloskop dargestellten Kurven abgelesen. Die Messung wurde bei $F = 22000\unit{\hertz}$ aus technischen Gründen vorzeitig abgebrochen.
Gleichzeitig wird die Generatorspannungsamplitude abgelesen um zu versichern dass diese konstant bleibt.
\subsection{Messung zur Verifikation der Integratorfunktion}
Nun werden verschiedene Spannunsformen vom Spannungsgenerator erzeugt, die generierte Spannung wird gemeinsam mit der durch den Tiefpass integrierten Spannung auf dem Oszilloskop angezeigt,
die angezeigten Bilder werden Fotografiert. Es werden die Spannungen für eine Sinus- eine Dreieck- und eine Reckteckspannung fotografiert.


\section{Auswertung}
\label{sec:Auswertung}


Die aus der auf dem Oszillator angezeigten Entladekurve entnommenen Messwerte werden in \ref{tab:tabelle1} dargestellt.

\begin{table}
  \centering
  \caption{Messwerte zur Entladekurve}
  \label{tab:tabelle1}
  \sisetup{table-format=1.1, per-mode=reciprocal}
  \begin{tblr}{
      colspec = {S[table-format=3.0] S[table-format=2.1] S},
      row{1} = {guard, mode=math},
    }
    \toprule
    t \mathbin{/} \unit{\second} \cdot 10^-3 [\pm 0.2]& U_c \mathbin{/} \unit{\volt} [\pm 0.4] & \\
    \midrule
    0   & 13.6 \\
    0.4 & 11   \\
    0.6 & 10   \\
    0.8 & 9.2  \\
    1.0 & 8.4  \\
    1.2 & 7.6  \\
    1.4 & 6.8  \\
    1.6 & 5.0  \\
    2.5 & 4.0  \\
    3.0 & 3.0  \\
    3.7 & 2.0  \\
    4.0 & 1.6  \\
    4.4 & 1.2  \\
    4.8 & 1.0  \\
    5.0 & 0.9  \\
    5.2 & 0.8  \\
    6.0 & 0.4  \\
    6.4 & 0    \\
    \bottomrule
  \end{tblr}
\end{table}

Nun wird in Abbildung \ref{fig:plot1} eine Fit-funktion mit polyfit \cite{numpy} erstellt, gefittet wird eine Funktion der gestalt:\\
\\
$\log{\frac{U_c}{U_0}} = a \cdot t + b$\\
\\
Die Parameter egeben sich zu:\\
$a = -561.333 ± 11.666$\\
$b = 1.389 ± 0.038$\\
\\
Um die Parameterberechung und das plotten zu ermöglichen wurde der letzte Messwert aus der Tabelle ignoriert.
\newpage

\begin{figure}
  \centering
  \includegraphics[width = 10cm]{plot1.pdf}
  \caption{Lineare Regression zur Bestimung der Zeitkonstante mithilfe der Entladungskurve}
  \label{fig:plot1}
\end{figure}

Da $a = -\frac{1}{RC}$, ist $RC = 0.00178\pm 0.00004$.



Eine weitere Methode die zeitkonstante RC zu bestimmen kann ausgeführt werden durch die Messung von der Amplitude unter variierender Frequenz der Wechselspannung.
Die Messwerte sind in Tabelle \ref{tab:tabelle2} aufgeführt.
\begin{table}
  \centering
  \caption{Messwerte zur Amplitude und Frequenz}
  \label{tab:tabelle2}
  \sisetup{table-format=1.1, per-mode=reciprocal}
  \begin{tblr}{
      colspec = {S[table-format=3.3] S[table-format=6.6] S},
      row{1} = {guard, mode=math},
      vline{2} = {2}{-}{text=\clap{$\pm$}},
    }
    \toprule
    \SetCell[c=2]{c} A \mathbin{/} \unit{\volt} & & F \mathbin{/} \unit{\hertz} [\pm 1] & \\
    \midrule
    6.8   & 0.4   &  50   \\
    4.8   & 0.4   &  100  \\
    1.2   & 0.4   &  500  \\
    0.6   & 0.2   &  1000 \\
    0.2   & 0.04  &  3000 \\
    0.12  & 0.04  &  5000 \\
    0.08  & 0.04  &  7000 \\
    0.06  & 0.02  &  10000\\
    0.05  & 0.02  &  12000\\
    0.04  & 0.01  &  15000\\
    0.035 & 0.01  &  17000\\
    0.034 & 0.004 &  19000\\
    0.032 & 0.004 &  20000\\
    0.03  & 0.002 &  22000\\
    \bottomrule
  \end{tblr}
\end{table}

\newpage

Anhand der Messwerte wird eine Ausgleichsfunktion der Gestalt:\\
\\
$U_c = \frac{1}{\sqrt(1+(F\cdot2\pi)\cdot a^2)}$ siehe \ref{eqn:amplitude2}\\
\\
mit curve-fit \cite{scipy} erstellt. $U_c$ ist dabei die Realtivamplitude $A/U_0$\\
\\
Die Parameter ergeben sich zu:\\
\\
$a = 0.00155 \pm 0.00007$\\
\\
Mit \ref{eqn:amplitude2} gilt also $RC = 0.00155 \pm 0.00007 $\\

Zum Zweck des Vergleichens wurde dieselbe Funktion noch einmal mit $RC = 0.00178$ geplottet. Der Graph ist zu sehen in Abbildung \ref{fig:plot2}.

\begin{figure}
  \centering
  \includegraphics[width = 10cm]{plot2.pdf}
  \caption{Fit der Messwerte der Relativamplitude und Frequenz im vergleich mit Graphen mit $RC = 0.00178$}
  \label{fig:plot2}
\end{figure}

\newpage





Für eine dritte Variante der Berechnung von RC wird der Phasenunterschied zwischen der Generatorspannung und der Kondensatorspannung abhängig von der Frequenz gemessen.
Die Messeregebnisses finden sich in Tabelle \ref{tab:tabelle3}.

\begin{table}
  \centering
  \caption{Messwerte zur Phasenverschiebung und Frequenz}
  \label{tab:tabelle3}
  \sisetup{table-format=1.1, per-mode=reciprocal}
  \begin{tblr}{
      colspec = {S[table-format=3.3] S[table-format=3.3] S[table-format=5.5]},
      row{1} = {guard, mode=math},
      vline{2} = {2}{-}{text=\clap{$\pm$}},
    }
    \toprule
    \SetCell[c=2]{c} \varphi \mathbin{/} rad & & F \mathbin{/} \unit{\hertz} [\pm 1] & \\
    \midrule
    3.77  &  0.15  &  50   \\
    2.2   &  0.13  &  100  \\
    1.6   &  0.6   &  500  \\
    1.3   &  0.6   &  1000 \\
    1.9   &  0.8   &  3000 \\
    1.9   &  0.6   &  5000 \\
    1.5   &  0.4   &  7000 \\
    1,9   &  0.6   &  10000\\
    1.5   &  0.8   &  12000\\
    1.9   &  0.9   &  15000\\
    1.7   &  0.4   &  17000\\
    1.4   &  0.5   &  19000\\
    1.5   &  0.5   &  20000\\
    1.38  &  0.28  &  22000\\
    \bottomrule
  \end{tblr}
\end{table}

\newpage

Im folgenden Plot \ref{fig:plot3} werden die Messergebnisse dargestellt, zum vergleich sind auch die Kurven der eigentlich erwarteten $\arctan$ Funktion \ref{eqn:phase1} mit den beiden vorher errechneten RC werten abgebildet.


\begin{figure}
  \centering
  \includegraphics[width = 10cm]{plot3.pdf}
  \caption{Messwerte im Vergleich mit erwarteter Funktion}
  \label{fig:plot3}
\end{figure}

\newpage

Nun werden die Messwerte für die relative Amplitude $A_{(\omega)}/U_0$ abhängig von der Phase dargestellt in einem Polarplot.
Zum Vergleich wird ausserdem die eigentliche Funktion geplottet, die die gestalt $A/U_0 = \frac{\sin(\varphi)}{\tan(\varphi)}$ hat.

\begin{figure}
\centering
\includegraphics[width = 10cm]{plot4.pdf}
\caption{Polarplot der relativen Amplitude abhängig von der Phasenverschiebung}
\label{fig:plot4}
\end{figure}

\newpage

Zur Verifikation der Integratorfunktion werden nun 3 Abbildungen \ref{fig:int1},  \ref{fig:int2} und \ref{fig:int3} gezeigt, die jeweils eine generierte Spannung gemeinsam mit der vom Tiefpass integrierten Spannung abbilden.

\begin{figure}
  \centering
  \includegraphics[width = 10cm]{V353foto1.pdf}
  \caption{Sinusspannung Integriert}
  \label{fig:int1}
\end{figure}
\begin{figure}
  \centering
  \includegraphics[width = 10cm]{V353foto2.pdf}
  \caption{Dreiecksspannung Integriert}
  \label{fig:int2}
\end{figure}
\begin{figure}
  \centering
  \includegraphics[width = 10cm]{V353foto3.pdf}
  \caption{Rechtecksspannung Integriert}
  \label{fig:int3}
\end{figure}










\section{Diskussion}
\label{sec:Diskussion}
\subsection{Bestimmung von RC mithilfe der Entladekurve}

Die Messwerte die aus der Entladekurve abgelesen wurden, haben einen relativ kleinen Fehler, da dieser über die ganze Messung gleich bleibt, fällt er bei den niedrigeren Werten 
stärker ins Gewicht. Die Messwerte unterliegen aufgrund des Ablesens von Hand und mit Augenmaß einer natürlichen Ungenauigkeit. Im Graphen der Messwerte ist zu erkennen, dass sie jedoch ziemlich nah an der Trendlinie liegen.
Aus den sich aus dem aus dem Fit ergebenden Parametern lässt sich problemlos die Zeitkonstante $RC = 0.00178\pm0.00004$ bestimmen.

\subsection{Bestimmung von RC mithilfe der relativen Amplitude und der Frequenz}

Auch bei dieser Messreihe ließ sich problemlos $RC = 0.00155\pm0.00007$ bestimmen, auch wenn die Messwerte hier einer erhöhten Unsicherheit unterliegen, da dass Ablesen am gerät durch immer kleinere Werte schwieriger und ungenauer wurde.
Aus dem Fit ließ sich problemos der RC-wert bestimmen, auch wenn die Kurve die Messwerte nur mit weniger guter Genauigkeit approximiert. Im Vergleich mit dem Graphen derselben Funktion aber mit dem vorher errechneten $RC = 0.00178\pm0.00004$
lässt sich erkennen dass es im Grunde keine Abweichung zwischen den beiden Kurven gibt die sich nicht durch einfache Messfehler und ungenauigkeiten erklären ließe. Eine fehlerquelle könnte außerdem sein, dass der Innenwiderstand nicht mit eingerechnet wurde, dieser könnte die Amplitude beeinflussen.

\subsection{Bestimmung von RC mithilfe der Phasenverschiebung und der Frequenz}

Die Messwerte der Phasenverschiebung abhängig von der Frequenz haben sehr große Fehler, ausserdem entsprechen sie im geplotteten Graphen nicht dem erwareteten Verlauf von \ref{eqn:phase1} und \ref{eqn:amplitude1}, die ersten beiden Messwerte sind sehr weit von dem Graphen der Funktion entfernt und haben auch ein gegensätzliches Verhalten, die anderen messwerte sind so ungenau und verstreut, dass insgesamt eine RC-berechnung hier nicht möglich war,
mögliche Gründe für die starken Abweichungen und das fehlerhafte Verhalten könnten die extrem ungenauen Ablesemethoden sein. Das ablesen auf dem Oszilloskop war extrem schwierig, aufgrund der kleinen Werte und den wechselnden skalen, außerdem haben sich die Kurven mit der Zeit ab und zu verschoben, und das genaue Ablesen wurde dadurch erschwert. Andere schwierigkeiten waren technischer Natur, so ging das Oszilloskop mehrfach einfach aus, oder die Kurven sprangen hin und her, nach einer gewissen Zeit, und bei hohen frequenzen hat das Oszilloskop überhaupt nichts mehr angezeigt beziehungsweise nur sporadische anscheinend zufällige Kurven.
Ein weiterer Grund könnte sein, dass vorallem im Bereich großer Frequenzen gemessen wurde, bei denen die skalen nicht mehr ausgereicht haben um genaue Ergebnisse abzulesen.
Es wäre auch möglich, dass die ersten messwerte nicht von Extremum zu Extremum abgelesen wurden, das könnte eine hohe Abweichung hervorrufen.

\subsection{Verifikation der Integratorfunktion des RC-Kreises}

Da die Werte der Phasenverschiebung so unzuverlässig und nahezu nicht zu gebrauchen sind, ist auch auf dem polarplot nichts verwertbares erkennbar, im Vergleich mit dem dargestellten Graphen lässt sich kaum übereinstimmung erkennen. Die am Oszilloskop aufgenommenen Bilder zeigen allerdings relativ eindeutig dass die eingegebene Spannung
vom Tiefpass integriert wird.






\printbibliography{}

Die verwendeten plots wurden mithilfe von matplotlib \cite{matplotlib} erstellt, die berechnungen wurden mit Python-numpy \cite{numpy}, Python-Scipy \cite{scipy} und die fehlerrechnung wurde mit Python-uncertainties \cite{uncertainties} gemacht.

\section{Anhang}
\begin{figure}
\centering
\includegraphics[width = 10cm]{V353ori1.pdf}
\caption{Original Messwerte seite 1}
\end{figure}
\begin{figure}
  \centering
  \includegraphics[width = 10cm]{V353ori2.pdf}
  \caption{Original Messwerte seite 2}
  \end{figure}
\end{document}
