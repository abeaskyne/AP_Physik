\section{Auswertung}
\label{sec:Auswertung}


In der nachfolgenden Tabelle \ref{tab:tabelle1} sind die gemessenen Werte der Stromstärke $I$ und des Abstandes $r$ sowie die aus $I$ nach \ref{} berechnete magnetische Flussdichte $B$ dargestellt.
Wobei $\mu_0 = 1.2566370621219 \cdot 10 ^ -6 $ gilt \cite{Formelsammlung}.
\begin{table}
  \centering
  \caption{Messwerte der Stromstärke, der magnetischen Flussdichte und des Abstandes r}
  \label{tab:tabelle1}
  \sisetup{table-format=1.1, per-mode=reciprocal}
  \begin{tblr}{
      colspec = {S[table-format=3.0] S[table-format=2.1] S},
      row{1} = {guard, mode=math},
    }
    \toprule
    r \mathbin{/} \unit{\centimeter} [\pm 0.1mm]& I \mathbin{/} \unit{\ampere} [\pm 0.1 A] & B \mathbin{/} \unit{\tesla} [\pm 0.00014 T] & \\
    \midrule
    10.35 & 2.7  & 0.00366 \\
    9.95  & 2.6  & 0.00353 \\
    8.62  & 2.3  & 0.00312 \\
    8.29  & 2.0  & 0.00271 \\
    6.35  & 1.8  & 0.00244 \\
    5.78  & 1.6  & 0.00217 \\
    5.35  & 1.5  & 0.00203 \\
    4.9   & 1.4  & 0.00190 \\
    4.5   & 1.35 & 0.00183 \\
    4.05  & 1.3  & 0.00176 \\
    \bottomrule
  \end{tblr}
\end{table}


In der folgenden Tabelle \ref{tab:tabelle2} werden die Messwerte für $I$, das daraus berechnete $B$ und die Periodendauer $T$ aufgeführt
\begin{table}
  \centering
  \caption{Messwerte der Stromstärke, der magnetischen Flussdichte und des Abstandes r}
  \label{tab:tabelle2}
  \sisetup{table-format=1.1, per-mode=reciprocal}
  \begin{tblr}{
      colspec = {S[table-format=3.0] S[table-format=2.1] S},
      row{1} = {guard, mode=math},
    }
    \toprule
    I \mathbin{/} \unit{\ampere} [\pm 0.1 A] & B \mathbin{/} \unit{\tesla} [\pm 0.00014 T] & T \mathbin{/} \unit{\second} \\
    \midrule
    0.5  & 0.00068  & 3.154 \\
    0.7  & 0.00095  & 2.557 \\
    0.9  & 0.00122  & 2.006 \\
    1    & 0.00136  & 1.948 \\
    1.3  & 0.00176  & 1.625 \\
    1.5  & 0.00203  & 1.518 \\
    1.8  & 0.00244  & 1.380 \\
    2.3  & 0.00312  & 1.174 \\
    3    & 0.00407  & 1.047 \\
    3.5  & 0.00475  & 0.892 \\
    \bottomrule
  \end{tblr}
\end{table}


