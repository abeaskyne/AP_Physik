\section{Theorie}
\subsection{Allgemeine Grundlagen und homogenes Magnetfeld eines Helmholzspulenpaares}
Die kleinste Objekt welches ein Magnetfeld erzeugt ist ein magnetischer Dipol, anders als bei elektrischen Feldern bei denen auch Monopole existieren. In der Praxis können
diese Dipole als Permanentmagneten oder Stromdurchflossene Leiter dargestellt werden. Für letztere lässt sich das magnetische Moment leicht berechnen.
Für das Magnetische Moment von Permanentmagneten existiert jedoch kein einfacher mathematischer Zusammenhang zur Berechnung. Stattdessen kann man sich eine Eigenschaft des Magnetischen moments 
zur experimentellen Bestimmung zu nutzen machen. In homogenen Magnetischen Feldern wirkt ein Drehmoment \\

\begin{centering}

   $ \vec{D} = \vec{\mu} \times \vec{B}$ %% kreuzprodukt und begriffe als vektor 

\end{centering}.
\\
Dabei dreht sich der Dipol bis $\mu$ und $B$ gleichgerichtet sind.
zur reallen Umsetzung eines homogenen Magnetischen Feldes können in der Praxis zwei in gleicher Richtung vom gleichen Strom durchflossene Spulen genutzt werden. Der Abstand 
der Spulen entspricht dabei ihrem Radius R. Das Feld in der Symmetrieachse der Spulen ist dann homogen und es ergibt sich dann mit Bio-Savart zu \\

\begin{centering}

$B(x)= \frac{\mu_{0}I}{2}\frac{R^2}{(R^2 + x^2)^{(5/2)}}$

\end{centering}.\\

Da in Praxis der Abstand der Spulen meist geringfügige Abweichungen vom Spulen Radius hat, folgt hier für das Feld in der Mitte der Spulen\\
\begin{equation}
 B(0)=\frac{\mu_{0}IR^2}{(R^2 + x^2)^{(3/2)}}
 \label{eqn:Bio1}
\end{equation}\\
für hinreichend  kleine Abweichungen von SPulenabstand d und Radius R gilt dann, das das Magnetfeld auf der Symmetrieachse Näherungsweise homogen angenommen werden kann.
Im folgenden werden die theoretischen Grundlagen zu diesen experimentellen Bestimmungsweisen erläutert.
\subsection{Bestimmung über Gravitation}
Auf eine mit dem Permanentmagneten verbundene Masse wirkt die Gravitationskraft, welche eine Drehmoment verursacht. Dem entgegen wirkt das magnetische Feld und das
dadurch entstehende Drehmoment. Für eine bestimmte Feldstärke B baut sich dabei ein Gleichgewicht auf. Es folgt daraus
\begin{equation}
    \mu_{Dipol} \cdot B = m \cdot r \cdot g 
    \label{eqn:grav2} 
\end{equation}.

\subsection{Bestimmung über Schwingunsdauer}
In einem homogenen Magntefeld (z.B. im Feld einer Helmholzspule) verhält sich der magnetische Dipol in der Kugel bei durch Auslenkung aus der Ruhelage erzeugter Schwingung
wie ein harmonischer Oszillator. \\

\begin{centering}

$-\lvert{\vec{\mu}_{\symup{Dipol}} \times \vec{B}}\rvert = J_{\symup{K}} \cdot \frac{d^2 \theta}{dt^2}$

\end{centering}
\\
beschreibt dann die Bewegungsgleichung des Problems. Als Lösung folgt dann die Beziehung\\

\begin{equation}
    \mu_{Dipol}=\frac{4\pi^2 J_{K}}{T^2 B}
    \label{eqn:schwingung1}
\end{equation}
\\


\subsection{Bestimmung durch Präzession}
Wenn auf die Drehachse eines in sich in Rotation befindlichen Körpers eine (auslenkende) Kraft wirkt, stellt sich eine Bewegung der Drehachse um den Drehimpulsvektor ein. 
Diese Art der Bewegung heißt Präzession und entsteht bei der Billardkugel durch eine Auslenkung bei Rotation, wobei die Rotation selbst für eine Stabilität der Bewegung sorgt.
Für die Bewegung der Kugel gilt dann\\

\begin{centering}

$\vec{\mu}_{Dipol} \times \vec{B}= \frac{d\vec{L}_{K}}{dt}$

\end{centering}

%% PRäzessionsfrequenz vllt einbauen
durch Ersetzung des Drehimpuls $L_{K}$ durch das Trägheitsmoment $J_{K}$ und durch die Umlaufzeit $T_{p}$ folgt dann 
\begin{equation}
    \frac{1}{T_{p}}= \frac{\mu_{Dipol} B}{2L_{K}\pi}
    \label{eqn:prz1}
\end{equation}







\label{sec:Theorie}

