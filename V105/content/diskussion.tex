\section{Diskussion}
\label{sec:Diskussion}

\subsection{Magnetisches Moment aus Gravitation}

Es lässt sich erkennen, dass die Messwerte sich ziemlich nah an der Ausgleichsgeradne halten. Auch der Achsenabschnitt ist in Vergleich mit der Steigung vernachlässigbar,
was auf eine gute Näherung schliessen lässt. Die Fehler der Parameter sind nicht alarmierend groß, dementsprechend ist auch die Unsicherheit des magnetischen Momentes klein,
Auftretende Fehler lassen sich dadurch erklären, dass der Aufbau aufgrund von Anfangsmomenten, ungenauigkeiten und der begrenzten Zeit nie exakt im Gleichgewicht war, sowie diversen ungenauigkeiten bei den Messungen und dem Ablesen von Werten.
Außerdem war der Aluminiumstab während der gesamten Durchführung bereits verbogen beziehungsweise kurz davor auseinander zu brechen, was er schließlich auch tat.

\subsection{Magnetisches Moment aus Oszillation}

Hier lässt sich erkennen, dass die Messwerte ebenfalls ziemlich nah an der Ausgleichsgeradne liegen, allerdings weisst die Regression einen nicht zu vernachlässgienden Achsenabschnitt von $b = (158.221 \pm 30.484) s^2$ auf, ausserdem ist das magnetische Moment um den Faktor $1.726$ kleiner als bei der ersten Methode.
Die (verglichen mit der Gravitationsmethode) relativ hohen Ungenauigkeiten (der hohe Achsenabschnittswert) lassen sich durch den Umstand erklären, dass bei dieser Methode, mehr von Hand gemessen wurde,
die zehnfache Periode wurde mit einer Stoppuhr gemessen die von Hand gestartet und gestoppt wurde, außerdem war die Anfangsauslenkung, da diese auch von Hand und mit Augenmaß eingestellt wurde, nicht immer konstant, und könnte stark variiert haben. Auch jegliche Effekte die die periodendauer während der messung der zehn
Zeiten verändert haben könnten, wie zum Beispiel Reibung, wurden in den berechnungen außer Acht gelassen.

\subsection{Magnetisches Moment aus Präzession}

Bei dieser Methode lässt sich an den Werten und der linearen regression erkennen, dass die Messwerte nicht sehr nah an der Ausgleichgeraden liegen. Dies führt zu hohen Unsicherheiten der Parameter und dementsprechend auch zu hohen unsicherheiten bei dem errechneten Wert
für das magnetische Moment. Der berechnete Wert für $\mu_0$ ist außerdem kleiner als bei den beiden vorherigen Methoden.
Diese ungenauigkeiten entstehen wahrscheinlich erneut durch die Durchführung des gesamten Vorganges von Hand, das Drehen der Kugel, das erkennen der passenden Frequenz, das einschalten des Stromes der das Magnetfeld erzeugt, die zeitmessung, die Beobachtung wann eine periode durchlaufen wurde und sämtliches Ablesen beziehungsweise Messen wurde von Hand und mit Augenmaß
gemacht. Bei so vielen fehlerquellen, kann man die Messwerte kaum als zuverlässig annehmen, um dem entgegenzuwirken wurde bei jeder Stromstärke drei mal gemessen, aber auch zwischen diesen Messergebnissen zeigen sich teilweise starke Schwankungen.

\subsection{Vergleich der magnetischen Momente}#

Die magnetischen momente sind:\\

\begin{centering}

$\mu_{\symup{g}} = (0.442 \pm 0.023) Am^2$\\
$\mu_{\symup{o}} = (0.256 \pm 0.013) Am^2$\\
$\mu_{\symup{p}} = (0.193 \pm 0.032) Am^2$\\

\end{centering}

$\mu_{\symup{g}}$ ist um den Faktor $1.726$ größer als $\mu_{\symup{o}}$\\
$\mu_{\symup{o}}$ ist um den Faktor $1.326$ größer als $\mu_{\symup{p}}$\\
Und $\mu_{\symup{g}}$ ist um den Faktor $2.290$ größer als $\mu_{\symup{p}}$\\

Die hohen Abweichungen der magnetischen Momente untereinander lassen sich durch die oben aufgeführten Fehlerquellen erklären.
Am genausesten scheint die Gravitationsmethode, da dort am wenigsten von Hand gemacht wird, und die Fehler der Messwerte relativ klein sind.